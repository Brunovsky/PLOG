% This is samplepaper.tex, a sample chapter demonstrating the
% LLNCS macro package for Springer Computer Science proceedings;
% Version 2.20 of 2017/10/04
%
\documentclass[runningheads]{llncs}
%
\usepackage{graphicx}
\usepackage{amsmath,mathtools,listings,inconsolata}
% Used for displaying a sample figure. If possible, figure files should
% be included in EPS format.
%
% If you use the hyperref package, please uncomment the following line
% to display URLs in blue roman font according to Springer's eBook style:
% \renewcommand\UrlFont{\color{blue}\rmfamily}

\newcommand*{\cc}[2]{\multicolumn{1}{#1}{#2}}

\begin{document}
%
\title{Doors}
%
%\titlerunning{Abbreviated paper title}
% If the paper title is too long for the running head, you can set
% an abbreviated paper title here
%
\author{Bruno Carvalho up201606517 \and Amadeu Pereira up201605646}
%
%\authorrunning{F. Author et al.}
% First names are abbreviated in the running head.
% If there are more than two authors, 'et al.' is used.
%
\institute{Faculdade de Engenharia da Universidade do Porto}		%\email{\{abc,lncs\}@uni-heidelberg.de}

\maketitle

\begin{abstract}
In this paper we describe and explore a simple restraint logic programming solution for the logic problem \textit{Doors}.

\keywords{doors \and logic puzzle \and rooms}
\end{abstract}

\section{Examples}

\begin{equation*}
\arraycolsep=4.0pt\def\arraystretch{1.3}
\begin{array}[c]{|c c c c|}
\hline
4 & \cc{ c|}{2} & \cc{|c }{4} & 2\\
\cline{2-2}\cline{4-4}
5 & \cc{ c }{3} & \cc{ c|}{5} & 3\\
\cline{2-2}
4 & \cc{ c|}{3} & \cc{|c }{ } & 4\\
\cline{1-1}\cline{3-3}
2 & \cc{ c|}{3} & \cc{|c }{2} & 4\\
\hline
\end{array}
\end{equation*}

\section{Problem Description}
\label{sec:description}

The logic problem \textit{Doors} is a puzzle on a rectangular board whose cells are either empty or contain natural numbers. The board is thought of like a \textit{house}. Each cell is a \textit{room}, and two adjacent cells are separated by a \textit{wall} with one \textit{door}. That door may be either open or closed. If it is open, then the cell can \textsl{see} its adjacent room through the doorway. A man standing in a room can look in all four directions --- north, east, west, south --- and count the number of \textit{visible rooms}.

The puzzle consists in discovering an assignment of open and closed doors to the walls of the board such that the natural number in each non-empty cell is how many rooms are visible from that cell (including itself). There may be multiple solutions, or none at all.

\section{Representation}
\label{sec:representation}

A puzzle of size $n\times m$ is represented internally by three matrices (list of lists): \textsl{Board}, of size $n\times m$, holding the cell numbers; \textsl{Vertical}, of size $n\times (m-1)$, holding the vertical walls; and \textsl{Horizontal}, of size $(n-1)\times m$, holding the horizontal walls.

For each cell $(R,C), R=1,2,\cdots,n, C=1,2,\cdots,m$, indices in \textsl{Board}, the left wall has index $(R,C-1)$ in \textsl{Vertical}, the right wall has index $(R,C)$ in \textsl{Vertical}, the top wall has index $(R-1,C)$ in \textsl{Horizontal}, the bottom wall has index $(R,C)$ in \textsl{Horizontal}.
Each wall in the board is assigned the number 0 for closed door and 1 for open door.

\section{Restrictions}
\label{subsec:solutionrestrictions}

When solving a puzzle \textsl{Board} is fully instantiated, while \textsl{Vertical} and \textsl{Horizontal} contain domain variables (domain $\{0,1\}$). Empty cells in the \textsl{Board} are represented by a $0$, as it is never a valid visible room counter.

\begin{equation*}
\arraycolsep=4.0pt\def\arraystretch{1.3}
\begin{array}[c]{|c c c c c c c c|}
\hline
1 & \cc{ c }{T} & \cc{ c }{?} & \cc{ c }{?} & \cc{ c }{?} & \cc{ c }{?} & \cc{ c }{?} & 2\\
%
X & \cc{ c }{A} & \cc{ c }{B} & \cc{ c }{C} & \cc{ c }{D} & \cc{ c }{E} & \cc{ c }{F} & G\\
%
? & \cc{ c }{M} & \cc{ c }{?} & \cc{ c }{?} & \cc{ c }{?} & \cc{ c }{?} & \cc{ c }{?} & ?\\
%
? & \cc{ c }{N} & \cc{ c }{?} & \cc{ c }{?} & \cc{ c }{?} & \cc{ c }{?} & \cc{ c }{?} & ?\\
\hline
\end{array}
\end{equation*}

Consider the puzzle above. The horizontal range $A-G$ consists of $7$ rooms and $6$ vertical doors: let $\{b,c,d,e,f,g\}$ be these vertical doors, from left to right.

Focus on room $A$.
If $b=0$ then $A$ sees no rooms to its right. If $b=1$ and $c=0$ then $A$ sees only room $B$.
A general formula can be deduced by noticing that closed doors behave as zero elements.

Let $e_A$ be the total number of rooms $A$ sees to its right (east), then

\begin{equation}\label{eq:acc}
e_A=b+b(c+c(d+d(e+e(f+f(g+g\cdot 0)))))
\end{equation}

Now, if we analogously define $w_A$ for west, $n_A$ for north and $s_A$ for south, then we find that the number in cell $A$ must be $e_A+w_A+n_A+s_A+1$.

Implementing these restrictions in \texttt{PROLOG} is surprisingly simple. We start with a predicate to compute formula (\ref{eq:acc}):

\begin{center}
\begin{minipage}{0.45\textwidth}
\centering\ttfamily
\begin{lstlisting}[language=Prolog]
calculate_value([], 0).
calculate_value([H|T], V) :-
    calculate_value(T, V1),
    V #= H + H*V1.
\end{lstlisting}
\end{minipage}
\end{center}

Then, for each non-zero cell $(R,C)$ on the \textsl{Board}, we retrieve as a list the four ranges of doors to the right, left, top and bottom of $(R,C)$, apply the formula for each list, and finally the restriction:

\begin{center}
\begin{minipage}{0.85\textwidth}
\centering\ttfamily
\begin{lstlisting}[language=Prolog]
restrict_cell(Board, _, _, [R,C]) :-
    matrixnth1([R,C], Board, 0), !. % empty cell
restrict_cell(Board, Vertical, Horizontal, [R,C]) :-
    matrixnth1([R,C], Board, Value),
    right_total(Vertical, [R,C], Right),
    left_total(Vertical, [R,C], Left),
    top_total(Horizontal, [R,C], Top),
    bot_total(Horizontal, [R,C], Bot),
    Right + Left + Top + Bot + 1 #= Value.
\end{lstlisting}
\end{minipage}
\end{center}


%
% ---- Bibliography ----
%
% BibTeX users should specify bibliography style 'splncs04'.
% References will then be sorted and formatted in the correct style.
%
% \bibliographystyle{splncs04}
% \bibliography{mybibliography}
%
\begin{thebibliography}{8}
	\bibitem{ref_article1}
	Author, F.: Article title. Journal \textbf{2}(5), 99--110 (2016)
	
	\bibitem{ref_lncs1}
	Author, F., Author, S.: Title of a proceedings paper. In: Editor,
	F., Editor, S. (eds.) CONFERENCE 2016, LNCS, vol. 9999, pp. 1--13.
	Springer, Heidelberg (2016). \doi{10.10007/1234567890}
	
	\bibitem{ref_book1}
	Author, F., Author, S., Author, T.: Book title. 2nd edn. Publisher,
	Location (1999)
	
	\bibitem{ref_proc1}
	Author, A.-B.: Contribution title. In: 9th International Proceedings
	on Proceedings, pp. 1--2. Publisher, Location (2010)
	
	\bibitem{ref_url1}
	LNCS Homepage, \url{http://www.springer.com/lncs}. Last accessed 4
	Oct 2017
\end{thebibliography}
\end{document}
